\chapter{一元二次方程}
\section{概念}
\subsection{定义}
只含有一个未知数,未知数的最高次数是~2,且系数不为~0,这样的方程叫一元二次方程.
\subsection{一般形式}
一元二次方程的一般形式为$ax^2+bx+c=0$($a$,$b$,$c$是已知数$a\neq 0 $).其中$a$,$b$,$c$分别叫做二次项系数、一次项系数、常数项.\par
(1)二次项、二次项系数、一次项、一次项系数,常数项都包括它前面的符号.\par
(2)要准确找出一个一元二次方程的二次项系数、一次项系数和常数项,必须把它先化为一般形式.\par
(3)形如$ax^2+bx+c=0$不一定是一元二次方程,当且仅当$a\neq 0$时是一元二次方程。\par
\subsection{判断一元二次方程}
一元二次方程必须同时满足以下三点:(1)方程是整式方程.(2)它只含有一个未知数.(3)未知数的最高次数是(同时还要注意在判断时,需将方程化成一般形式).
\section{解方程}
\textbf{1.直接开平方法:}对形如$(x+a)^2=b(b\geq 0)$的方程两边直接开平方而转化为两个一元一次方程的方法. \par 
(1)$9x^2-16=0$ \hspace{1cm}(2)$(x+5)^2-16=0$ \hspace{1cm}(3)$(x-5)^2=(3x+1)$  \\
\vspace{2cm}

\textbf{2.配方法1:}解一元二次方程时,在方程的左边加上一次项系数一半的平方,再减去这个数,使得含未知数的项在一个完全平方式里,这种方法叫做配方,配方后就可以用因式分解法或直接开平方法了,这样解一元二次方程的方法叫做配方法(\textbf{注意}:用配方法解一元二次方程$x^2+px+q=0$,当对方程的左边配方时,一定记住在方程的左边加上一次项系数的一半的平方后,还要再减去这个数).\\
(1)$x^2+6x-5=0$\hspace{1cm} (2)$x^2-\frac{7}{2}-2=0$\\
\vspace{2cm}

\textbf{3.配方法2:}当一元二次方程的形式为$ax^2+bx+c=0(a\neq 0,a\neq 1 )$时,用配方法解一元二次方程的步骤:(1)先把二次项的系数化为1:方程的左、右两边同时除以二项的系数;(2) 移项:在方程的左边加上一次项系数的一半的平方,再减去这个数,把原方程化为$(x+m)^2=n $的形式;
(3)若$n\geq 0 $,用直接开平方法或因式分解法解变形后的方程。\\
(1)$3x^2-9x+2=0 $  \hspace{3cm}  (2)$-x^2-4x+3=0$\\
\vspace{2cm}

\textbf{4.求根公式法:}一元二次方程$ax^2+bx+c=0(a\neq 0 ) $的求根公式是:$x=\frac{-b\pm \sqrt{b^2-4ac} }{2a} $
用求根公式法解一元二次方程的步骤是:(1)把方程化为$ax^2+bx+c=0(a\neq 0) $的形式,确定的值$a,b,c $(注意符号);(2)求出$b^2-4ac $的值;(3)若$b^2-4ac\geq 0$ ,则a,b把及$b^2-4ac $的值代人求根公式,求出 $x=\frac{-b\pm \sqrt{b^2-4ac} }{2a} $,求出$x_1,x_2$.\\
(1)$2x^2-3x-1=0$ \hfill  (2)$2x(x+\sqrt{2})+1=0$ \hfill  (3)$x^2+x+25=0$\\
\vspace{3cm}

\textbf{5.因式分解法:} 把方程左边的多项式(方程右边为0时)的公因式提出,将多项式写出因式的乘积形式,然后利用“若$pq=0$时,则$p=0$或$q=0$”来解一元二次方程的方法,称为因式分解法。\\
(1)$x^2-5x+6=0$  \hfill  (2)$x^2-x-12=0$\par
\vspace{2cm}

(3)$x^2-4x+3=0$ \hfill (4)$2x^2-3x-5=0$\par
\vspace{2cm}

\textbf{列}选方法解方程:\\
(1)$(2x-3)^2=9(2x+3)^2$ \hfill (2)$x^2-8x+6=0$ \hfill  (3)$(x+2)(x-1)=0$
\vspace{2cm}

\section{一元二次方程根的判别式} 
(1)$\bigtriangleup =b^2-4ac >0 \longmapsto $ 方程有两个不相等的实数根.\\
(2)$\bigtriangleup =b^2-4ac =0 \longmapsto $
方程有两个相等的实数根.\\
(3)$\bigtriangleup =b^2-4ac <0 \longmapsto $
方程有无实数根.\\
\textbf{列} 判断下列一元二次方程根的情况:\\
(1)$2x^2-3x-5=0 $\hfill (2)$9x^2=30x-25 $\hfill (3)$x^2+6x+10=0$\\
\vspace{2cm}

\textbf{例:} $m$为何时,方程$(2m+1)x^2+4mx+2m-3=0$的根满足下列情况:\\
(1)有两个不相等的实数根;\hfill (2)有两个不相等的实数根;\hfill (3)没有实数根;\\
\vspace{2cm}

\section{根与系数的关系}
若$x_1,x_2$是一元二次方程$ax^2+bx+c=0(a\neq 0 )$的两个根,则有,$x_1+x_2=-\frac{b}{a},x_1x_2=\frac{c}{a} $根据一元二次方程的根与系数的关系求值常用的转化关系:\\
${x_1}^2+{x_2}^2=(x_1+x_2)^2-2x_1x_2$ \\
$(x_1-x_2)^2=(x_1+x_2)^2-4x_1x_2$\\
${x_1}^2\cdot x_2+x_1\cdot {x_2}^2=x_1\cdot x_2(x_1+x_2)$ \\
$(x_1+a)(x_2)=x_1\cdot x_2+a(x_1+x_2)+a^2$ \\
$\frac{1}{x_1}+\frac{1}{x_2}=\frac{x_1+x_2}{x_1\cdot x_2}$ \\
$\frac{1}{{x_1}^2}+\frac{1}{{x_2}^2}=\frac{{x_1}^2+{x_2}^2}{{x_1}^2\cdot {x_2}^2}=\frac{(x_1+x_2)^2-2x_1x_2}{(x_1\cdot x_2)^2}$ \\
$|x_1-x_2|=\sqrt{(x_1-x_2)^2}=\sqrt{(x_1+x_2)^2-4x_1 x_2}$ \\
\textbf{例}:已知方程$2x^2-5x-3=0$的两根为$x_1,x_2$,不解方程,求下列各式的值.\par
(1)${x_1}^2+{x_2}^2$ \hspace{5cm} (2)$(x_1-x_2)^2$
\section{习题}
\subsection{定义}
1.关于$x$的一元二次方程$(k-1)x^2-2x+1=0$有两个不相等的实数根,则实数$k$的取值范围是.\par
2.已知关于x的一元二次方程$x^2+x+m^2-2m=0$有一个实数根为-1,求m的值及方程的另一实根.
3.将下列方程化为一般形式,并分别指出它们的二次项系数、一次项系数和常数项.

(1)$(x-2)(x+3)=8$  \hspace{3cm}    (2)$(3x-4)(x+3)=(x+2)^2$\\
\vspace{2cm}

4.已知关于$x$的方程$(m-1)x^{m^2+2}-(m+1)x-2=0$是一元二次方程时,则$m$= \\
5.关于x的方程$(k+1)x^2+3(k-2)x+k^2-42=0$的一次项系数是-6,则$k$=\\
\subsection{解一元二次方程}
1.用配方法解一元二次方程$x^2-6x-4=0$,下列变形正确的是\\
A.$(x-6)^2=-4+36$\hfill B.$(x-6)^2=4+36$\hfill C.$(x-3)^2=-4+9$\hfill D.$(x-3)^2=4+9$ \hfill  

2.若方程$x^2-x=0$的两个根为$x_1,x_2(x_1<x_2)$,则$x_2-x_1$=

3.方程$x^3-x=0$的解是

4.方程$(2x+1)(x-1)=8(9-x)-1$的根为

5.用合适的方法解方程:\\
$x(x-2)+x-2=0$  \hspace{2cm}   $2x^2-7x-2=0$ \hspace{2cm}  $2(x-1)^2+5(x-1)+2=0$\\
\vspace{2cm}

$\sqrt{2x^2-9x+5}=x-3 $  \hspace{2cm}  $x^3-2x^2-3x=0$  \hspace{2cm} \\ 

\subsection{根与系数的关系}
1.已知关于x的一元二次方程$x^2+mx-8=0$的一个实数根为2,则另一实数根及m的值分别为\\
A.4,-2  \hfill   B.-4,-2  \hfill  C.4,2 \hfill D.-4,2\par
2.设$x_1,x_2$是方程$x^2+5x-3=0$的两根,则${x_2}^2+{x_2}^2$的值是\\
A.19 \hfill  B.25   \hfill  C.31  \hfill  D.30  \par

3、若$x_1,x_2$是一元二次方程$x^2-2x-1=0$的两个根,则${x_1}^2-x_1+x_2$的值为\\
A.-1 \hfill B.0      \hfill     C.2  \hfill      D.3 \par
4.已知关于x的一元二次方程$x^2+mx+n=0$的两个实数根分别为$x_1=-2,x_2=4$,则m+n的值是\\
 	A.-10 \hfill B.10 \hfill	C.-6	\hfill D.2  \par
5.设$x_1,x_2$是方程$2x^2+14x-16=0$的两个实数根,则$\frac{x_2}{x_1}+\frac{x_1}{x_2} $的值为\par
6.设$a,b$,是方程$x^2+x-2013=0$的两个不相等的实数根,$a^2+2a+b$的值\\
7.已知一元二次方程$x^2+3x+1=0$的两个根为$x_1,x_2$ 那么$(1+x_1)(1+x_2)$的值等于\\
8.已知为$x_1,x_2$ 是方程$x^2-3x+1=0$的两个实数根,则$\frac{1}{x_1}+\frac{1}{x_2} $的值是\\
9.关于$x$的一元二次方程$x^2+(2k+1)x+k^2+1=0$有两个不等实根$x_1,x_2$.\\
(1)求实数$k$的取值范围.\\
\vspace{2cm}

(2)若方程两实根$x_1,x_2$ 满足$|x_1|+|x_2|=x_1•x_2$,求k的值\\
\vspace{2cm}

\subsection{根的判别式}
1.一元二次方程$(x+1)^2-2(x-1)^2=7$的根的情况是(  )\\
A.无实数根  \hfill B.有一正根一负根 \hfill C.有两个正根	 \hfill D.有两个负根\\
2.下列选项中,能使关于$x$的一元二次方程$ax^2-4x+c=0$一定有实数根的是\\
A.a>0 \hfill B.a=0 \hfill C.c>0 \hfill D.c=0\\
3.若关于$x$的一元二次方程$4x^2-4x+c=0$有两个相等实数根,则c的值是\\
 	A.-1 \hfill	B.1\hfill C.-4  \hfill D.4\\
4.若矩形的长和宽是方程程$2x^2-16x+m=0(0<m\leq 32)$的两根,则矩形的周长为\\
5.关于x的一元二次方程$ax^2 +2x+1=0$的两个根同号,则a的取值范围是\\
6.已知关于x的一元二次方程$\frac{1}{2} mx^2+mx+m-1=0$有两个相等的实数根.\\
(1)求m的值;\\
\vspace{2cm}

(2)解原方程;\\
\vspace{2cm}

7.已知:关于$x$的方程$x^2+2mx+m^2-1=0$.\\
(1)不解方程,判别方程根的情况;\\
\hspace{2cm}

(2)若方程有一个根为3,求m的值;
\hspace{2cm}

\section{一元二次方程的应用}
\begin{itemize}
\item 审题
\item 设未知数
\item 列方程
\item 解方程
\item 检验根是否符合实际情况
\item 作答
\end{itemize}
\subsection{传播问题}
1.有一人患了流感,经过两轮传染后共有~121~人患了流感,每轮传染中平均一个人传染了几个人?\par
\vspace{2cm}

2.某种植物的主干长出若干数目的支干,每个支干又长出同样数目的小分支,主干、支干和小分支的总数是91,每个支干长出多少小分支.\par
\vspace{2cm}

3.参加一次足球联赛的每两队之间都进行一场比赛,共比赛45场比赛,共有多少个队参加比赛.\par
\vspace{2cm}

4.参加一次足球联赛的每两队之间都进行两次比赛,共比赛90场比赛,共有多少个队参加比赛.\par
\vspace{2cm}

5.生物兴趣小组的学生,将自己收集的标本向本组其他成员各赠送一件,全组共互赠了182件,这个小组共有多少名同学.\par
\vspace{2cm}

6.一个小组有若干人,新年互送贺卡,若全组共送贺卡72张,这个小组共有多少人.\par
\vspace{2cm}

\subsection{平均增长率问题}
1.青山村种的水稻2001年平均每公顷产7200公斤,2003年平均每公顷产8450公斤,求水稻每公顷产量的年平均增长率.\par
\vspace{2cm}

2.某种商品经过两次连续降价,每件售价由原来的90元降到了40元,求平均每次降价率是多少.\par
\vspace{2cm}

3.某种商品,原价50元,受金融危机影响,1月份降价$10\%$,从2月份开始涨价,2月份的售价为64.8元,求2、3月份价格的平均增长率。\par
\vspace{2cm}

4.某药品经两次降价,零售价降为原来的一半,已知两次降价的百分率相同,求每次降价的百分率.\par
\vspace{2cm}

5.为了绿化校园,某中学在2007年植树400棵,计划到2009年底使这三年的植树总数达到1324棵,求该校植树平均每年增长的百分数.\par
\vspace{2cm}

\subsection{握手问题}
1.一个小组有若干人,新年互送贺卡,已知全组共送贺卡56张,则这个小组有多少人.\par
\vspace{2cm}

2.假设每一位参加宴会的人见面时都要与其他人握手致意,这次宴会共握手28次,问参加这次宴会的共有多少人.\par
\vspace{2cm}

3.参加一次聚会的每两个人都握了一次手,所有人共握手10次,有多少人参加聚会.\par
\vspace{2cm}

4.参加一次足球联赛的每两个队之间都进行两次比赛,共要比赛90场,共有多少个队参加比赛.\par
\vspace{2cm}

5.学校组织一次兵乓球比赛,参赛的每两个选手都要比赛一场,所有比赛一共有36场,问有多少名同学参赛?用一元二次方程,化成一般形式.\par
\vspace{2cm}

\subsection{商品销售问题}
1.某商店购进一种商品,进价30元.试销中发现这种商品每天的销售量P(件)与每件的销售价X(元)满足关系:P=100-2X销售量P,若商店每天销售这种商品要获得200元的利润,那么每件商品的售价应定为多少元?每天要售出这种商品多少件.\par
\vspace{2cm}

2.某玩具厂计划生产一种玩具熊猫,每日最高产量为40只,且每日产出的产品全部售出,已知生产$ⅹ$只熊猫的成本为$R$(元),售价每只为$P$(元),且$R$与x的关系式分别为$R=500+30X$,$P=170—2X$.\par

(1).当日产量为多少时每日获得的利润为$1750$元.\par
\vspace{2cm}

(2).若可获得的最大利润为$1950$元,问日产量应为多少.\par
\vspace{2cm}

3.某水果批发商场经销一种高档水果,如果每千克盈利10元,每天可售出500千克,经市场调查发现,在进货价不变的情况下,若每千克涨价1元,日销售量将减少20千克。现该商品要保证每天盈利6000元,同时又要使顾客得到实惠,那么每千克应涨价多少元.\par
\vspace{2cm}

4.服装柜在销售中发现某品牌童装平均每天可售出20件,每件盈利40元。为了迎接“六一”儿童节,商场决定采取适当的降价措施,扩大销售量,增加盈利,减少库存。经市场调查发现,如果每件童装每降价4元,那么平均每天就可多售出8件。要想平均每天在销售这种童装上盈利1200元,那么每件童装应降价多少元.\par
\vspace{2cm}

5.西瓜经营户以$2$元/千克的价格购进一批小型西瓜,以3元/千克的价格出售,每天可售出200千克。为了促销,该经营户决定降价销售。经调查发现,这种小型西瓜每降价0.1元/千克,每天可多售出40千克。另外,每天的房租等固定成本共24元。该经营户要想每天盈利200元,应将每千克小型西瓜的售价降低多少元\par


\subsection{面积问题}
1.一个直角三角形的两条直角边的和是14cm,面积是$24cm^2$,求两条直角边的长。\par
\vspace{2cm}

2.一个直角三角形的两条直角边相差5cm,面积是7$cm^2$,求斜边的长。\par
\vspace{2cm}

3.一个菱形两条对角线长的和是10cm,面积是12$cm^2$,求菱形的周长(结果保留小数点后一位)\par
\vspace{2cm}

4.为了绿化学校,需移植草皮到操场,若矩形操场的长比宽多14米,面积是3200平方米则操场的长为()米,宽为()米。\par
\vspace{2cm}

5.若把一个正方形的一边增加2cm,另一边增加1cm,得到的矩形面积的2倍比正方形的面积多11$cm^2$,则原正方形的边长为.\par
\vspace{2cm}

6.一张桌子的桌面长为6米,宽为4米,台布面积是桌面面积的2倍,如果将台布铺在桌子上,各边垂下的长度相同,求这块台布的长和宽。\par
\vspace{2cm}

7.有一面积为54$cm^2$的长方形,将它的一组对边剪短5cm,另一组对边剪短2cm,刚好变成一个正方形,这个正方形的边长是多少?\par
\vspace{2cm}

\subsection{数字问题}
1.两个数的和为8,积为9.75,求这两个数。\par
\vspace{2cm}
 
2.两个连续偶数的积是168,则这两个偶数是.\par
\vspace{2cm}

3.一个两位数,个位数字与十位数字之和为5,把个位数字与十位数字对调,所得的两位数与原来的两位数的乘积为736,求原来的两位数。\par
\vspace{2cm}
 









